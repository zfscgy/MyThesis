\chapter{绪论}
\section{课题背景}
一直以来,贷款业务就是商业银行的核心业务之一。但是贷款业务伴随着信用风险。如果贷款者没有按时偿还贷款,银行就会承担经济损失。因此如何尽可能保证贷款者的可靠性,减少坏账的发生率,是银行急需解决的问题之一。同样的需求也存在于提供贷款服务的互联网金融公司。

近年来,随着计算机性能的提高,深度学习在各个领域的应用愈加广泛。在图像识别、机器翻译、模式识别等领域,深度学习的模型效果都达到了业界的顶尖水准\cite{lecun2015deep}。

深度学习可以看做是一个强大的函数拟合工具,特别适用于普通的统计学方法难以处理的超高维数据。对于风险控制的场景来说,数据就是商业银行或互联网金融公司所得到的用户画像,标签就是用户违约的概率。已有的用户画像及其是否违约的记录,就是训练数据。


\section{研究现状}
\subsection{传统的风控方法}
传统的风险控制方法,包括银行工作人员对贷款者的情况进行主观分析,从而判断其偿还能力。银行一般使用的是"The Five Cs of Credit"\cite[122]{apostolik2009foundations},从如下五个维度判断贷款者的信用:
\begin{itemize}
	\item Character: 贷款者的“人格”,包含其名声、能力等。
	\item Capital: 贷款者的资本状况,包含其公司的运营情况、现金流量、杠杆水平、固定资产等。很显然资本状况好的贷款者更有可能按时偿还贷款。
	\item Condition: 包含了各方面的情况,比如贷款者所在区域的整体经济情况,贷款者所在行业的发展状况。
	\item Capacity: 判断贷款公司是否有足够的规模产生足够偿还贷款的现金流。
	\item Collateral: 抵押品,判断公司是否有足够价值的抵押品可供违约的时候进行偿还。
\end{itemize}
同时银行也可以通过客观的数据来判断贷款者的信用情况,比如通过具体的数据建立模型,判断贷款者的还款意愿。同时在贷款已经发放之后,银行还会追踪贷款者的各方面行为,包括:
\begin{itemize}
	\item 对贷款者信用状况的重新评估
	\item 对贷款使用的监控,防止贷款者将贷款用于不良的目的
	\item 对贷款者经营活动的监督
\end{itemize}
\subsection{基于统计学以及机器学习的风控方法}
贷款者一般都有很多可量化的客观指标,比如贷款者的年龄、性别、收入、家庭人口数量、贷款金额、贷款期限等。国外已经有大量研究采用不同模型对其进行拟合,从而预估贷款者的违约可能性。比如: 
\begin{itemize}
	\item 通过主成分分析,将贷款者通过聚类划为数个分组然后再分别进行线性拟合\cite{2016konocreditrisk}
	\item 用企业数十种的财务数据,通过神经网络判断其债务违约概率\cite{angelini2008neural}
\end{itemize}
\subsection{基于大数据的风控方法}
之前研究所使用的数据集往往规模较小,每个贷款者仅有几个到几十个特征。对于个人贷款,商业银行或互联网公司就很难搜集到足够的与贷款直接相关的财务信息。但是在大数据发达的今天,贷款提供方可以通过多种途径获得贷款者的其他信息,比如互联网金融公司的软件可以通过用户的手机权限,合法地获得用户的通话记录、地理位置等,从这些信息里可以提取大量用户的特征,以此建立模型。