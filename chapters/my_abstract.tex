\begin{abstract}
随着深度学习的兴起,以神经网络、随机森林为代表一系列深度学习模型被广泛应用在各个领域,在模式识别、图像分类、自然语言处理等领域都取得了非常强大的效果。

对信用风险的控制一直是银行等金融机构的迫切需求。互联网的发展和大数据的兴起,金融机构能够取得前所未有的大量用户数据,由于数据的规模空前,难以人工考察,传统的风险控制方法对此无能为力。因此很多金融机构开始探寻把深度学习技术应用于金融风控,获得了较好的效果。但是由于各种风控报告缺乏统一的格式和接口,目前并没有一套通用的特征提取系统,因此不同的特征提取方式在模型效果上会有显著的差异。同时数据中存在大量缺失值,使得模型训练、预测和解释更为困难。

本文从一组金融数据集出发,探究了深度学习应用于风控的方法。本文具体阐述了数据处理、模型搭建、效果评估的一系列流程,并且提出了一种基于特征嵌入的缺失值填补办法以应对金融数据集的数据缺失问题。本文通过更加细致的特征工程,使得对用户违约预测的效果达到了较好的水平,同时也能较准确地补全原始数据中的缺失值,并且提高某些模型的预测效果。

\keywords{深度学习;大数据;金融;风险控制;信用风险;缺失值}
\end{abstract}

\begin{enabstract}
With the rising of deep learning, a series of models represented by neural networks and random forests are widely used in various fields, and proofed to be very powerful in fields like pattern reconginition, image classfication and natural language processing.

It has always been an eager demand for financial facilities like banks to control credit risk. As we are in the era of internet and big data, those financial facilities have gathered an unprecedental large amount of users' data. Tt's impossible to mannually analyse those data, so traditional risk management methods are failing on this job. Thus many financial facilities are exploring to apply deep learning methods to risk management. Because various risk reports lack of a unified format and interface, there is no universal system for feaure-extraction. Therefore, different ways of feature-extraction will result in significant diffrence in models' performance. And massive missing values existing in raw data causes models' training, predicting and explaning become more difficult.

This thesis uses a combination of financial database to study for the application of deep learning in risk management, and also describes major steps in the process including data extraction, model building and assesment of models. And in this thesis I proposed a new mehod for missing data imputation in order to handle the data missing problem commonly exists in financial datasets. This thesis improves models' performance on predicting user default by more careful feature engineering, and complements raw data's missing values with a high accuracy, thus improves performance in some models.

\enkeywords{Deep Learning; Big Data; Finance; Risk Management; Credit Risk; Missing Data}
\end{enabstract}