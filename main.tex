% !TeX encoding = UTF-8
% !TeX program = xelatex
% !TeX spellcheck = en_US

\documentclass[bachelor]{ustcthesis}
% doctor|master|bachelor [academic|professional] [chinese|english] [print|pdf]
% [super|numebers|authoryear]

\title{基于深度学习的金融风控研究 }
\author{郑非}
\major{计算机科学与技术}
\supervisor{郑小林\ 副教授}
\cosupervisor{李金龙\ 副教授}
% \date{二〇一七年五月一日} % 注释掉则为今日
% \professionaltype{专业学位类型}
% \secretlevel{秘密}        % 绝密|机密|秘密,注释本行则不保密
% \secretyear{20}           % 保密年限

\entitle{Deep Learning in Fiancial Risk Management}
\enauthor{Fei Zheng}
\enmajor{Computer Science and Technology}
\ensupervisor{Asso. Prof. Xiaolin Zheng}
\encosupervisor{Asso. Jinlong Li}
% \endate{May 1, 2017}      % Today if commented
% \enprofessionaltype{Professional degree type}
% \ensecretlevel{Secret}    % Top secret|Highly secret|Secret


% 加载宏包和配置
\usepackage{graphicx}
\graphicspath{{figures/}}
\usepackage{booktabs}
\usepackage{longtable}
\usepackage[ruled,linesnumbered]{algorithm2e}
\usepackage{siunitx}
\usepackage{amsthm}
\usepackage{hyperref}

\usepackage{xcolor}
\usepackage{listings}
\renewcommand{\lstlistingname}{代码}
\lstset{ %  
  backgroundcolor=\color{white},   % choose the background color; you must add \usepackage{color} or \usepackage{xcolor}  
  basicstyle=\footnotesize,        % the size of the fonts that are used for the code  
  breakatwhitespace=false,         % sets if automatic breaks should only happen at whitespace  
  breaklines=true,                 % sets automatic line breaking  
  captionpos=bl,                    % sets the caption-position to bottom  
  commentstyle=\color{green},    % comment style  
  deletekeywords={...},            % if you want to delete keywords from the given language  
  escapeinside={\%*}{*)},          % if you want to add LaTeX within your code  
  extendedchars=true,              % lets you use non-ASCII characters; for 8-bits encodings only, does not work with UTF-8  
  frame=single,                    % adds a frame around the code  
  keepspaces=true,                 % keeps spaces in text, useful for keeping indentation of code (possibly needs columns=flexible)  
 %  keywordstyle=\color{blue},       % keyword style  
  %language=Python,                 % the language of the code  
  morekeywords={*,...},            % if you want to add more keywords to the set  
  numbers=left,                    % where to put the line-numbers; possible values are (none, left, right)  
  numbersep=5pt,                   % how far the line-numbers are from the code  
  numberstyle=\tiny\color{gray}, % the style that is used for the line-numbers  
  % rulecolor=\color{black},         % if not set, the frame-color may be changed on line-breaks within not-black text (e.g. comments (green here))  
  showspaces=false,                % show spaces everywhere adding particular underscores; it overrides 'showstringspaces'  
  showstringspaces=false,          % underline spaces within strings only  
  showtabs=false,                  % show tabs within strings adding particular underscores  
  stepnumber=1,                    % the step between two line-numbers. If it's 1, each line will be numbered  
  % stringstyle=\color{orange},     % string literal style  
  tabsize=2,                       % sets default tabsize to 2 spaces  
  %title=myPython.py                   % show the filename of files included with \lstinputlisting; also try caption instead of title  
}  

\DeclareRobustCommand\cs[1]{\texttt{\char`\\#1}}
\newcommand\pkg{\textsf}

\renewcommand\vec{\symbf}
\newcommand\mat{\symbf}
\newcommand\ts{\symbfsf}
\newcommand\real{\mathbf{R}}




\begin{document}

% 研究生论文:
%   封面,原创性声明和授权使用声明
%   frontmatter: 摘要,目录,[图、表清单],[符号说明]
%   mainmatter: 正文章节,参考文献
%   appendix: 附录
%   backmatter: 致谢,已发表论文列表
%
% 本科生论文:
%   封面
%   frontmatter: 致谢,目录,摘要
%   mainmatter: 正文章节,参考文献
%   appendix: 附录

\maketitle
\makestatement

\frontmatter
%\input{chapters/abstract}
\tableofcontents
% \listoffigures
% \listoftables
%\input{chapters/notation}

\mainmatter
\begin{abstract}

\end{abstract}

\begin{enabstract}

\end{enabstract}
\chapter{绪论}
\section{课题背景}
一直以来,贷款业务就是商业银行的核心业务之一。但是贷款业务伴随着信用风险。如果贷款者没有按时偿还贷款,银行就会承担经济损失。因此如何尽可能保证贷款者的可靠性,减少坏账的发生率,是银行急需解决的问题之一。同样的需求也存在于提供贷款服务的互联网金融公司。

近年来,随着计算机性能的提高,深度学习在各个领域的应用愈加广泛。在图像识别、机器翻译、模式识别等领域,深度学习的模型效果都达到了业界的顶尖水准\cite{lecun2015deep}。

深度学习可以看做是一个强大的函数拟合工具,特别适用于普通的统计学方法难以处理的超高维数据。对于风险控制的场景来说,数据就是商业银行或互联网金融公司所得到的用户画像,标签就是用户违约的概率。已有的用户画像及其是否违约的记录,就是训练数据。


\section{研究现状}
\subsection{传统的风控方法}
传统的风险控制方法,包括银行工作人员对贷款者的情况进行主观分析,从而判断其偿还能力。银行一般使用的是"The Five Cs of Credit"\cite[122]{apostolik2009foundations},从如下五个维度判断贷款者的信用:
\begin{itemize}
	\item Character: 贷款者的“人格”,包含其名声、能力等。
	\item Capital: 贷款者的资本状况,包含其公司的运营情况、现金流量、杠杆水平、固定资产等。很显然资本状况好的贷款者更有可能按时偿还贷款。
	\item Condition: 包含了各方面的情况,比如贷款者所在区域的整体经济情况,贷款者所在行业的发展状况。
	\item Capacity: 判断贷款公司是否有足够的规模产生足够偿还贷款的现金流。
	\item Collateral: 抵押品,判断公司是否有足够价值的抵押品可供违约的时候进行偿还。
\end{itemize}
同时银行也可以通过客观的数据来判断贷款者的信用情况,比如通过具体的数据建立模型,判断贷款者的还款意愿。同时在贷款已经发放之后,银行还会追踪贷款者的各方面行为,包括:
\begin{itemize}
	\item 对贷款者信用状况的重新评估
	\item 对贷款使用的监控,防止贷款者将贷款用于不良的目的
	\item 对贷款者经营活动的监督
\end{itemize}
\subsection{基于统计学以及机器学习的风控方法}
贷款者一般都有很多可量化的客观指标,比如贷款者的年龄、性别、收入、家庭人口数量、贷款金额、贷款期限等。国外已经有大量研究采用不同模型对其进行拟合,从而预估贷款者的违约可能性。比如: 
\begin{itemize}
	\item 通过主成分分析,将贷款者通过聚类划为数个分组然后再分别进行线性拟合\cite{2016konocreditrisk}
	\item 用企业数十种的财务数据,通过神经网络判断其债务违约概率\cite{angelini2008neural}
\end{itemize}
\subsection{基于大数据的风控方法}
之前研究所使用的数据集往往规模较小,每个贷款者仅有几个到几十个特征。对于个人贷款,商业银行或互联网公司就很难搜集到足够的与贷款直接相关的财务信息。但是在大数据发达的今天,贷款提供方可以通过多种途径获得贷款者的其他信息,比如互联网金融公司的软件可以通过用户的手机权限,合法地获得用户的通话记录、地理位置等,从这些信息里可以提取大量用户的特征,以此建立模型。
\chapter{相关理论综述}
本文所要完成的任务是根据用户的数据来预测用户的违约概率,因此是一个回归任务。下面介绍了几种最广泛使用的回归分析的模型,并且分析了其评估的效果和可行性,从中选择了三种模型作为本文主要采用的预测模型。
\section{问题描述}
我们的目标是给定一组用户的特征$ x \in R^m $,求出该用户的违约概率 $ y \in [0, 1]$。
为了统一描述,对于某个风控的数据集,我们把第j个用户的特征记为$\vec x_j = [x_{j,1}, x_{j,2}, ..., x_{j,n}]$, 因此第j个用户的第i个特征就是$x_{j,i}$。第j个用户的标签为$y_j$, 满足
\[
y_j =\begin{cases}  0 \quad \text{用户j的贷款按时偿还} \\ 1 \quad \text{用户j的贷款违约} \end{cases}
\]

我们希望可以找到函数$f(\vec x)$ 使得对于每个用户j,$f(\vec x_j)$ 都尽量接近 $y_j$, 因此可以用均方误差作为损失函数,即优化目标为 $\min \limits_{f} \sum_j (f(\vec x_j) - y_j)^2$。可以使用如下所述的多种回归模型对其拟合。

\section{回归模型}

\textbf{线性回归}:
线性回归也称作最小二乘法拟合,是最简单的回归方法,最小化目标为
\[\min \limits_{w_i, b} \sum_{j} (\vec w \cdot \vec x_j + b - y_j)^2\]
 
通过最小二乘法可以对$\vec w$和$b$进行估计。线性回归易于实现,并且具有良好的可解释性,拟合出来的$w_i$就是第i个特征的权重。

\textbf{逻辑回归}:
逻辑回归是一种常用的回归方法。用均方误差作为损失函数的逻辑回归的目标为:
\[\min \limits_{w_i, b} \sum_{j} (\text{sigmoid} ( \vec w\cdot \vec x_j + b) - y_j)^2\]

其中,$ \text{sigmoid}(x) = \dfrac{1}{1+e^{-x}} $。逻辑回归一般使用梯度下降法求解最佳参数。

逻辑回归和线性回归一样易于实现,并具有良好的可解释性。所不同的是逻辑回归拟合的值域是[0,1],更加适合二分类数据。

\textbf{支持向量机}:
支持向量机\cite{scholkopf2001svm}是一种强大的二分类算法,最小化目标为:
\[ \min \limits_{w_i, b} \dfrac{1}{n} \sum_{j}(\max(0, 1-y_j(\vec w \cdot x_j))) + \lambda ||w||^2  \]

其原理就是找到一个超平面能够将$y=0$和$y=1$的两类样本尽可能分离开。

支持向量机可以通过求解对偶问题把原问题转换为带约束的二次规划问题,然后通过多种方法求解。同时支持向量机也支持Kernel Tricks,即通过定义Kernel Function(核函数)把原数据映射到无穷维内积空间,然后进行拟合。常用的有高斯核等。支持向量机具有非常强大的二分类功能。

\textbf{人工神经网络}:
人工神经网络是一种从人类神经元连接方式启发而得到的机器学习模型,其每一层的基本方法是通过矩阵乘法将m维的数据映射到n维,然后再用激活函数产生输出,公式如下:
\[ \bm{y} = \text{activation}(W\bm{x} + \bm{b}) \]

神经网络可以有很多层,也可以有不同的结构,比如卷积神经网络、循环神经网络等。\cite{nndp} 神经网络具有强大的拟合能力,理论上足够大的单隐层神经网络可以逼近任意函数\cite{cybenko1989approximation}。神经网络在各种回归和分类任务上都具有良好的表现,但是容易出现过拟合问题,并且缺少可解释性。

\textbf{决策树}:
决策树\cite[311]{james2013introduction}是一种简单的模型,通过树结构来进行分类。决策树的每一个节点相当于一个if-else语句,对某个属性进行划分。对于该属性的值小于节点的值的样本,就把其送入左边的节点继续判断;反之则送入右边的节点。决策树把样本的向量空间分割为多个高维长方体,任何一个样本必然落在其中的一个长方体中。

决策树每一步从某一个叶节点出发,挑选一个属性进行分裂。一般来说分类决策树挑选分裂点的原则是信息增益,也就是让每个子节点的信息熵的和尽可能变小。也就是说其优化目标是:

\[ \min \sum_k \sum_m p_{mk}\log(p_{mk}) \]

其中, $p_{mk}$表示在叶子(区域)m上类别为k的样本的比例。

由于决策树的性质,理论上决策树可以对任何分布的样本进行分类。但是也很容易出现过拟合的现象,所以现在广泛采用大量浅层决策树的集成模型进行预测,如下面介绍的随机森林和XGBoost。而且决策树具有非常好的可解释性,因此被广泛应用于各种特征无关的分类工作。

\textbf{随机森林}:
随机森林\cite[320]{james2013introduction}是多个决策树的集成模型,通过Boostrap采样算法从原始数据集中生成多个新的数据集,并分别用其训练不同的决策树模型。其中决策树选择分裂的时候仅仅从所有属性里面的一个子集挑选属性,以此降低过拟合的问题。
随机森林能够显著提高决策树的效果,并且保留了决策树的可解释性。

\textbf{XGBoost}:
XGBoost\cite{chen2016xgboost}是另一种决策树的集成模型,可以用于回归和分类。对于某一样本,其输出就是XGBoost模型中的每一棵决策树对应的叶子节点的权重之和。XGBoost通过Boosting算法,在每轮迭代中加入一棵新的决策树来尽可能降低整体的损失函数。XGBoost通过限制单棵决策树的深度、部分采样以及对叶子和权重的规范化来减少过拟合。

XGBoost的算法通过预先索引、对属性的并行,部分采样等方式提高每棵树的生成速度,极大提高了运算效率。
\\

因为我们的问题是对用户是否违约进行预测,是一个二分类问题,因此直接使用线性回归的效果较差。支持向量机的运算开销过大,需要十几个小时才能做出一组预测,因此在之后的实验中没有采用该方法。同时决策树、随机森林和Xgboost呈现递进关系,Xgboost的一轮迭代就是决策树。因此在后文中我们采取了其中逻辑回归、人工神经网络和Xgboost这三个模型进行实验。

\section{特征处理方法}
如果原始数据的维度过高,或是存在一些缺失值,会影响模型的拟合效果,或是增加拟合的成本。一般可以通过主成分分析解决原始数据维度过高的问题,通过AutoEncoder去除原始数据中的缺失值、噪声等异常数据。

\textbf{主成分分析}:
主成分分析\cite{shlens2014pca} 是一种降维的方法,把样本空间从m维降到k维(k<m),满足
\[
\min \sum_{i = 1}^{n} ||\bm{x_i} - \sum_{j=1}^{k}(<\bm{x_i},\bm{\alpha_j}>\alpha_j)||^2
\]

即当样本从k维的“压缩状态”回到m维的时候,其损失(用差值的模平方衡量)之和最小。因此这是一种保留最多信息的线性映射。

\textbf{AutoEncoder}:
AutoEncoder是一种输入和输出的大小相同的神经网络,其原理是把N维的输入数据转换到m维的隐藏层(m<n),然后再转换成N维。AutoEncoder可以通过提取隐藏层来实现数据降维,也可以通过再设置一个与输入同样大小的输出层,对输入数据进行数据缺失值的填补。\cite{vincent2008autoencoder}

因为现在深度学习的任务都可以在GPU上运行,因此可以接受维度很高的数据,而主成分分析必然损失一部分信息。因此在之后的讨论中我们均使用原始数据,并未采用主成分分析对其进行降维。

\section{模型效果评估方法}
\subsection{训练集和测试集}
为了评估模型的预测效果,我们需要从原始数据中划分一部分数据集作为测试集,测试集不参与训练。因为各种深度学习模型都具有强大的拟合能力,所以模型在训练集上的误差会趋向于0,所以为了评估模型真实的拟合能力,我们需要让训练好的模型在测试集上做出预测,查看其效果。
\subsection{K折交叉验证}
随着现代计算机计算性能的提升,对于大数据集的计算也不再昂贵。因此现在一般采用K折交叉验证的方式评估模型性能。即,把数据集分成K等份,依次选择其中的K-1份作为训练集,剩下的一份作为验证集,分别计算其误差。通过这种方式,可以显著降低评估误差,降低某些偶然因素导致的评估的不准确性,找出最佳的模型。\cite[213]{james2013introduction}
\subsection{模型评估指标}
\subsubsection{准确率}
模型的输出结果如果大于等于0.5,就认为其预测的标签是1,反之则为0。则准确率计算公式如下: 
\[
Accuracy = \dfrac{\text{预测准确的样本数}}{\text{总样本数}}
\]
\subsubsection{AUC(Area Under Roc Curve)}
ROC曲线,根据预测样本的真阳性率和假阳性率形成的曲线。其计算公式为
\[
    AUC = \int_{fpr=0}^1 tpr \mathrm{d}tpr
\]
其中,fpr表示假阳性率(实际标签为0,预测标签为1的样本数量/实际标签为0的样本总数(正样本总数)),tpr表示真阳性率(实际标签为1,预测标签为1的样本数量/实际标签为1的样本总数(负样本总数))
我们令预测标签为1的概率门槛从1变到0,则真阳性率和假阳性率将会从0变到1。设当有i个假阳性样本时,则假阳性率为i/N,N为所有实际标签为0的样本。不妨设第i的假阳性样本是刚刚加入的,也就是说模型对其的输出是这i个里面最小的,计其为$p_i$,则此时对应的真样本的数目为$\sum\limits_{j \in \text{正样本}} 1_{f(j) > f(i)}$
其中,
\[
1_{x>y} =\begin{cases}  0 \quad x<y \\ 1 \quad x>y \end{cases}
\]
根据以上的推导,可以看出AUC可以用如下公式计算:
\[
    AUC = \sum\limits_{i \in \text{正样本} } \sum\limits_{j \in \text{负样本} }{1_{f(i)>f(j)}}
\]
随机预测的AUC的值在0.5左右,AUC的值越大,说明模型越倾向于使得正样本的输出大于负样本的输出。
\chapter{风控原始数据预处理}
本文中所使用的数据集来自某互联网公司的数个风控平台的数据库。因为原始数据的格式不统一,含有大量字符串、压缩数据、列表、JSON等数据,因此必须经过处理才可以作为模型输入。数据预处理的好坏会极大影响模型预测的效果\cite{kotsiantis2006data}。我们的处理目标就是把原始数据中可能有用的数据尽可能提取成数值,同时尽可能丢弃无效的数据,保留对模型预测有效的数据,返回一个 (用户数目$\times$单个用户特征数目大小) 的表。并且表中的每一项数值都介于0到1之间,有利于对于数值敏感的模型的训练。
\section{大数据平台Hadoop}
Hadoop\footnote{https://hadoop.apache.org}是一个开源的大数据平台,提供了一套分布式存储和分布式任务调度的框架,能够让服务器集群更加有效地处理超大规模的数据。Spark\footnote{https://spark.apache.org/docs/latest/}是一套快速通用的集群计算系统,能够执行大量数据处理的任务,并且提供了Python、Java和Scala语言的高级API。Spark可以通过Hadoop进行任务调度。
\section{数据格式统一}
\subsection{JSON数据的提取}
JSON\footnote{https://www.json.org}是一种轻量级的数据表示格式,可以表示key-value类型的对象,以及列表,支持嵌套。每个对象的值可以是对象、字符串、数字或者是布尔值以及null。下列代码是一个简单的JSON数据示例。
\begin{lstlisting}[language=PYTHON, caption={JSON示例}, label={JSON示例}]
{
	"姓名":"张三",
	"家属": 
	{
		"父亲":"张父",
		"母亲":"张母"
	},
	"出生日期":"1999-9-9",
	"学历记录":
	[
		{
			"中学":"张村中学",
			"成绩":"95"
		},
		{
			"大学":"中国科学技术大学",
			"成绩":"1.0"
		}
	]
}
\end{lstlisting}

对于JSON格式的数据,处理方式为:

递归地取出所有非列表结构的字段。为了在数据处理之后不丢失原有的属性的信息,按照属性名拼接的方式产生列名。比如对于上面的示例中的“父亲”一项,就把值保存在列名为"家属\_\_\_父亲"的列中。对于列表形式的数据,根据实际情况,一般列表当中的对象都是字典类型,所以把其中所有数值类型字段进行求和和平均,加入列明为属性名拼接加上“\_\_\_SUM”或"\_\_\_AVG"标识符的列中,比如上面示例里的“学历记录”可以产生两列名为“学历记录\_\_\_成绩\_\_\_SUM”和“学历记录\_\_\_成绩\_\_\_AVG”的记录。

\begin{algorithm}[htbp]
	\SetAlgoLined
	\KwData{Json}
	\KwResult{OutputDict}
	OutputDict = \{\}\\
	UnParsedJsonElems = [Json]\\
	Prefix = ""\\
	\While{UnParsedJsonElems is not empty}{
		elem = UnParsedJsonElems.pop(0)\\
		\For{attr in elem.stringAndNumericalAttributes}
		{
			OutputDict[Prefix + attr] = elem[attr]\\
		}
		\For{attr in elem.listAttributes}
		{
			\For{subAttr in elem[attr][0].numericalAttributes}
			{
				OutputDict[Prefix+"\_\_\_"+attr + "\_\_\" + subAttr + "\_\_\_SUM"] = SUM(elem[attr], subAttr) \\
				OutputDict[Prefix+"\_\_\_"+attr + "\_\_\" + subAttr + "\_\_\_AVG"] = AVG(elem[attr], subAttr) \\
			}
		}
		\For{attr in elem.objectAttributes}
		{
			UnparsedJsonElem.push(elem[attr])
		}
	}
	\caption{Json处理算法示例}
	\label{algo:algorithm1}
\end{algorithm}
\subsection{特大数量异构JSON数据融合}
由于实验的数据集中含有数十万条用户的JSON数据,而且通过样本得知每个样本的JSON数据都略有不同,比如用户2可能有用户1中不存在的字段,用户1也可能有用户2中不存在的字段。由于整体数据量过于庞大,如果要遍历所有的用户数据,由于各用户JSON字段的差别,无法并行地进行处理,因此效率极其低下。为了获取大部分用户都拥有的字段,我们采用采样的方法。

假设某一字段a, 在数据集的所有用户中出现的频率是p,那么缺失率就是1-p。在N次采样中,字段a都没有出现的概率为
\[ p_{\text{miss in N samples}} = (1-p)^N\]

因此如果取采样数目为80,那么对于在样本中出现频率仅为10\%的字段,其在采样中不出现的概率也仅仅不到$\frac{1}{5000}$,因此我们可以认为大小为80的采样样本中包含了几乎所有出现频率大于0.1的字段。

因此首先把采样中出现的字段存放在字典中,将每个字段对应到一个整数,可以大大减少因为字段过长产生的冗余内存。然后通过该字典把用户的JSON数据进行统一处理,如果用户含有不包含在字典中的JSON字段则丢弃,如果用户缺少字典中的JSON字段,则将该字段补为空值。同时把字段名都改为字典中对应字段的整数Value。

\subsection{多源数据整合}
数据集中的每个表都有一个唯一的用户ID标识,因此可以通过表连接的方式获得总表。
\subsection{数据的数值化转换}
由于不同的数据源使用不同的方法表示数据的缺失、异常值,比如有的采用字符串"无数据"表示,有的使用null值表示,有的使用空字符串表示,所以转换的时候需要注意,并且将其都转换为nan值。

为了获取每一列可能的非数值数据,通过遍历的方法,逐列进行检查,把每一列出现的非数值值记录到文件里并且人工检查之。尽可能把能够量化的字符串变量进行数值化转换。如对于“用户与律师通话频率”字段,有“频繁通话”、“正常通话”、“很少通话”、“从未通话”四种字符串,可以分别转换为1、0.667、0.333、0四种数值。

对于类别类型的数据,比如“省份”之类,采用one-hot编码,也就是说把列数从原来的一列扩充到与类别数量相同的列数。



\section{数据清洗}
由于通过前面步骤得到的数据量依然非常巨大,而大量对模型预测无效的数据会对模型产生不良影响,同时数据量过大也会极大地增加模型的训练和运行成本,所以我们需要对每一列数据进行初步分析,删除那些可能无效的列。

\subsection{数据的初步分析}
我们采用的方法是对每一列数据进行分析。
\begin{itemize}
	\item 缺失率:即该列数据在所有样本中缺失的比率。如果缺失率过高,则考虑丢弃该列。
	\item 最大值、最小值、标准差。这些数值可以反映该列数据的变动程度。如果最大值非常接近最小值或者标准差过小,则可认为该列数据在每一个样本上变动很小,没有反映出样本之间的差异性,考虑丢弃该列。
	\item 与标签的相关系数。如果与标签的相关系数很小,则说明该列很可能没有有用信息。但是也有可能存在该列数据仅仅和标签的一阶相关系数小,可能有其他的非线性关系或者和别的特征组合影响标签的情况。
\end{itemize}
我们通过以上三点综合考虑每一列数据的有效性,最终决定是否要丢弃该列数据。

为了使得最后的数据分布在0—1之间,把每一列都按照如下方式处理:

$normalized\_value = \dfrac{value - min}{max - min}$
%\input{chapters/intro}
%\input{chapters/floats}
%\input{chapters/math}
%\input{chapters/citations}
\bibliography{bib/ustc}
\appendix
%\input{chapters/complementary}

\backmatter
%\input{chapters/acknowledgements}
%\input{chapters/publications}

\end{document}
